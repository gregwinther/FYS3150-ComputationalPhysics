\documentclass[10pt, a4paper]{amsart}

\usepackage[]{graphicx}
\usepackage[]{hyperref}
\usepackage[]{physics}
\usepackage[]{listings}
\usepackage[utf8]{inputenc}
\usepackage[toc,page]{appendix}

\lstset{
	frame = single,
	language = C++,
	showstringspaces = false,
	tabsize = 2,
	otherkeywords = {self},
	keywordstyle = \color{blue},
	identifierstyle=\color{deepgreen},
 	stringstyle=\color{orange},
 	backgroundcolor=\color{mygray}
}

\title[Schrödinger's Equation in 3D]{Schrödinger's Equation in a Three-Dimensional Harmonic Oscillator Well \\
  \hrulefill\small{ FYS3150: Computational Physics }\hrulefill}
  
\begin{document}

\begin{titlepage}
\begin{abstract}
THIS IS NOT THE ABSTRACT FOR THIS PROJECT :)
A one-dimensional version of the Poisson equation with Dirichlet boundary conditions is solved using two different algorithms. The two algorithms employed are the tridiagonal matrix algorithm and the LU decomposition method. We find that a fine-tuned version of the tridiagonal matrix algorithm is $10^5$ times faster than the general LU decomposition method and $10$ times as precise.
\end{abstract}
\maketitle
\tableofcontents
\end{titlepage}

\section{Introduction}

The aim of this study is to solve Schrödinger's equation for two electrons in a three-dimensional  harmonic oscillator well with and without repulsive Coulomb interaction. The equation is reformulated to a discretized in order to be represented as a matrix eigenvalue problem that can be solved with Jacobi's method. 

\section{Theory}
We assume that electrons confined in a small area move in a three-dimensional harmonic oscillator potential and repel each other via the static Coulomb interaction. To simplify the problem, spherical symmetry is assumed. 

\subsection{A single electron}
The radial part of Schrödinger's equation for a single electron reads
\begin{equation}
\label{eq:schr1}
-\frac{\hbar^2}{2m}\left(\frac{1}{r^2}\frac{d}{dr}r^2\frac{d}{dr}-\frac{l(l+1)}{r^2} \right)R(r) + V(r)R(r) = ER(r)
\end{equation}
where $V(r)$ is the harmonic oscillator potential\footnote{One can in theory insert any potential into the Schrödinger equation, but in this case the harmonic oscillator potential is employed.}, $V(r)=\frac{1}{2}kr^2$ where $k=m\omega^2$, and consequently $E$ is the energy of the harmonic oscillator in three dimensions. The angular frequency of the oscillator is given by $\omega$ and is related to the energy of the system as follows
\begin{equation}
E_{nl}=\hbar\omega\left(2n + l  + \frac{3}{2} \right), \quad n,l \in \mathbb{N} 
\end{equation}
However, we will consider cases with zero angular momentum only ($l=0$). Moreover, by a somewhat lengthy substitutive exercise will culminate in a rewriting of Schrödinger's equation in \ref{eq:schr1} to a dimensionless form
\begin{equation}
\label{eq:schr2}
-\frac{d^2}{d\rho^2}u(\rho) + \rho^2u(\rho) = \lambda u(\rho)
\end{equation}
where $u(r) = R(r)/r$ and $\rho = r/\alpha$ is a dimensionless parameter for length. The constant $\alpha$ is $\alpha = \sqrt{\hbar^2/mk}$ and the energy parameter (eigenvalues) are $\lambda = 2m\alpha^2E/\hbar^2$. The three first eigenvalues will be $\lambda_0= 3$, $\lambda_1 = 7$ and $\lambda_0=11$.

The second derivative in equation \ref{eq:schr2} will be approximated by the well-known central finite difference method for the second derivative,
\begin{equation}
\frac{du^2}{d\rho^2}=\frac{u(\rho+h)-2u(\rho)+u(\rho-h)}{h^2}+\mathcal{O}(h^2)
\end{equation}
where $h = (\rho_N-\rho_0)/h$ is the step size. By denoting $u(\rho_0+ih)$ by $u_i$, equation \ref{eq:schr2} can be further rewritten to
\begin{align}
-\frac{1}{h^2}u_{i-1}+\left(\frac{2}{h^2} + \rho_i^2 \right)u_i-\frac{1}{h^2}u_{i+1}=\lambda u_i \label{eq:schr3}\\
A\vb{u} = \lambda \vb{u} \label{eq:mat1}
\end{align} 
where the matrix in \ref{eq:mat1} is
\begin{equation}
\begin{bmatrix}
\frac{1}{h^2} + \rho_i^2 & -\frac{1}{h^2} & 0 & 0 & 0 & \dots & 0 \\
-\frac{1}{h^2} & \frac{1}{h^2} + \rho_i^2 & -\frac{1}{h^2} & 0 & 0 & \dots & 0 \\
0 & -\frac{1}{h^2} & \frac{1}{h^2} + \rho_i^2 & -\frac{1}{h^2} & 0 & \dots & 0  \\
\vdots & \vdots & \vdots &\ddots & \ddots & \ddots & \vdots \\
0 & 0 & 0 & \dots &  -\frac{1}{h^2} & \frac{1}{h^2} + \rho_i^2 & -\frac{1}{h^2} \\
0 & 0 & 0 & 0 & \dots & -\frac{1}{h^2} & \frac{1}{h^2} + \rho_i^2 
\end{bmatrix}
\end{equation}

By finding the eigenvectors, or the energies, corresponding to the eigenvalues $\lambda_i$  equation \ref{eq:schr3} above, one has found the wave function that describe the behavior of the electron for a particular state $i$. To have something to compare against it is useful to state the exact, analytical wave function for the three lowest states. The general form of these radial wave functions is 
\begin{equation}
\label{eq:radialschr1}
R_{n,l=0}(r)=N_ne^{-\frac{m\omega}{2\hbar}r^2}\sqrt{L_n}\left(-\frac{m\omega}{2\hbar} r^2 \right)
\end{equation}
where $L_n$ refers to a particular Laguerre polynomial. More on the Laguerre polynomials can be found in appendix \ref{app:laguerre}. Reevaluating this function for the three lowest energy states and normalizing yields the following equations
\begin{align}
\abs{u_0(\rho)}^2 &= \frac{4}{\sqrt{\pi}}\rho^2e^{-\rho^2} \\
\abs{u_1(\rho)}^2 &= \frac{8}{3\sqrt{\pi}}\rho^2\left( \frac{3}{2}-\rho^2 \right)e^{-\rho^2} \\
\abs{u_2(\rho)}^2 &= \frac{8}{15\sqrt{\pi}}\rho^2\left( \frac{15}{4}-5\rho^2 +\rho^4\right)e^{-\rho^2}
\end{align}
It is only sensible to deal with the absolute squared values of the wave functions, because this is a probability density function for the position of the electron.

\subsection{Two electrons}
Now  there are two harmonic oscillating electrons in the well. At first these are assumed to be non-interacting. The radial part of the Schrödinger equation for this situation reads
\begin{equation}
\left(-\frac{\hbar^2}{2m}\frac{d^2}{dr_1^2} \right)
\end{equation}

\subsection{Preservation of orthogonality and dot product in unitary transform}

Consider a basis of vectors 
\begin{equation}
\vb{v}_i = \begin{bmatrix}
v_{i1} \\
\vdots \\
v_{in}
\end{bmatrix}
\end{equation}
We assume that the basis is orthogonal, that is
\begin{equation}
\vb{v}_j^T\vb{v}_i=\delta_{ij}.
\end{equation}
It can be shown that a unitary transformation $\vb{w}_i = U\vb{v}_i$, where $U$ is a unitary matrix such that $U^TU=I$, the dot product and orthogonality is preserved. 
\begin{align}
\vb{w}_i^T\vb{w}_j &= (U\vb{v}_i)^T  U\vb{v}_i= \vb{v}_i^TU^T U\vb{v}_j = \vb{v}_i^T I \vb{v}_j \nonumber \\
&\rightarrow \vb{w}_i^T\vb{w}_j = \vb{v}_i^T\vb{v}_j = \delta_{ij},
\end{align}
which means that both the dot product and orthogonality is preserved.

\begin{thebibliography}{9}

\bibitem{morten2}
	Hiorth-Jensen, M.,
	\emph{Project 2, Computational Physics I FYS3150/FYS4150},
	University of Oslo,
	2016.

\end{thebibliography}

\begin{appendix}
\section{Laguerre Polynomials}
\label{app:laguerre}
Generally, the name Laguerre polynomials is used for solutions to 
\begin{equation}
x\frac{d^2y}{dx^2}+(\alpha+1-x)\frac{dy}{dx} + ny = 0.
\end{equation}
These polynomials, usually denoted, $L_0$, $L_1$, $L_2$ etc are a polynomial sequence which may be defined by the Rodriguez formula
\begin{equation}
L_n(x) = \frac{e^x}{n!}\frac{d^n}{dx^n}(e^{-x}x^n)=\frac{1}{n!}\left(\frac{d}{dx} -1 \right)x^n
\end{equation}
The first few Laguerre polynomials are shown in table \ref{tab:laguerre}.

\begin{table}[ht]
	\centering
	\caption{The first few Laguerre polynomials}
	\begin{tabular}{cl} \hline
	$n$ & $L_n(x)$  \\ \hline
	$0$ & $1$ \\
	$1$ & $-x+1$ \\
	$2$ & $\frac{1}{2}(x^2-4x+2)$ \\
	$3$ & $\frac{1}{6}(-x^3+9x^2-18x+6)$ \\
	$4$ & $\frac{1}{24}(x^4-16x^3+72x^2-96x+24$ \\
	$5$ & $\frac{1}{120}(-x^5+25x^4-200x^3+600x^2-600x+120) $ \\
	$6$ & $\frac{1}{720}(x^6-36x^5+450x^4-2400x^3+5400x^2-4320x+720)$ \\ \hline
	\end{tabular}
	\label{tab:laguerre}
\end{table}

\end{appendix}

\end{document}