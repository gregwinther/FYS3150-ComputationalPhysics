\documentclass[10pt, a4paper]{amsart}

\usepackage[]{graphicx}
\usepackage[]{hyperref}
\usepackage[]{physics}

\usepackage[T1]{fontenc}

\title[Solving the Poisson-equation in one dimension]{Solving the Poisson-equation in one dimension \\
  \hrulefill\small{ FYS3150: Computational Physics }\hrulefill}

\author[Sebastian G. Winther-Larsen]{Sebastian G. Winther-Larsen \\
  \href{https://github.com/gregwinther/FYS3150/}{\texttt{github.com/gregwinther}}}

\begin{document}

\begin{titlepage}
\begin{abstract}
Lorem ipsum dolor sit amet, consectur adipiscing lit. Nullam ut lacus eget lorem...
\end{abstract}
\maketitle
\tableofcontents
\end{titlepage}

\section{Introduction}

\section{Theory}
\subsection{The Poisson Equation}

The Poisson equation is a classical equation from electromagnetism. The electrostatic potential $\Phi$ is generated by a localized charge distribution $\rho(\vb{r})$. In three dimensions the equation reads
\begin{equation}
\laplacian\Phi=-4\pi \rho(\vb{r})
\end{equation}
where $\laplacian$ is the Laplace operator. In three dimensions the Laplace operator can be expressed using spherical coordinates, but in this study I am assuming that $\Phi$ and $\rho$ are spherically symmetric, thus reducing the equation to a one-dimensional problem. only dependent on radius $r$.
\begin{equation}
\laplacian=\frac{1}{r^2}\frac{d}{dr}\left(r^2\frac{d\Phi}{dr} \right)
\end{equation}
By substituting $\Phi(r)=\phi(r)/r$ the Poisson equation is reduced to 
\begin{equation}
\frac{d^2\phi}{dr^2}=-4\pi r\rho(r)
\end{equation}
and by letting $\phi \rightarrow u$ and $r \rightarrow x$ one is left with the very simple equation
\begin{equation}
-u''(x)=f(x) \label{eq:2nd}
\end{equation}

The inhomogenous term $f$, or source term, is given by the charge distribution $\rho$ multiplied by $r$ and the constant $-4\pi$. In this study, however, the source term will be $f(x)=100e^{-10x}$ and the results can be compared to the analytical solution $u(x)=1-(1-e^{-10})x-e^{-10x}$. 


\subsection{Approximation of the Second Derivative}
In this study the one-dimensional Poisson equation will be solved with Dirichlet boundary conditions by rewriting it as a set of linear equations. The discretized approximation of $u$ is defined as $v_i$ with grid points $x_i=ih$, step size of $h=\frac{1}{n+1}$, in the interval $x_0=0$ to $x_{n+1}=1$ and with boundary conditions $v_0=v_n+1=0$. The interior solution $v_i \forall i \in {1,...,n}$ is to be found. The second order derivative is approximated with the three point formula such that equation \ref{eq:2nd} becomes
\begin{equation}
-\frac{v_{i+1}-2v_i+v{i-1}}{h²}=f_i \label{eq:2approx}
\end{equation}

By defining $\tilde{\vb{b}}=h^2f_i$ one can rewrite equation \ref{eq:2approx} as $-v_{i+1}-2v_i+v{i-1}=h^2f_i$. If we ignore the end points, $i=0$ and $i=n+1$, this equation can be represented as a matrix equation.
\begin{equation}
\begin{bmatrix}
2 & -1 & 0 & 0 & \cdots & 0 & 0 & 0 \\
-1 & 2 & -1 & 0 & \cdots & 0 & 0 & 0 \\
0 & -1 & 2 & -1 & \cdots & 0 & 0 & 0 \\ 
& & \vdots &  & \ddots &  & \vdots & \\
0 & 0 & 0 & 0 & \cdots & -1 & 2 & -1 \\
0 & 0 & 0 & 0 & \cdots & 0 & -1 & 2 
\end{bmatrix}
\begin{bmatrix}
v_1 \\
v_2 \\
v_3 \\
\vdots \\
v_{n-1} \\
v_n
\end{bmatrix}=
\begin{bmatrix}
b_1 \\
b_2 \\
b_3 \\
\vdots \\
b_{n-1} \\
b_n
\end{bmatrix}
\end{equation}

\end{document}
